\section{Summary in Dutch}

Een volwassen mens bestaat gemiddeld uit ongeveer 37 biljoen cellen, die allemaal oorspronkelijk afkomstig zijn van \'e\'en enkele eicel. Tijdens de embryonale ontwikkeling vermenigvuldigt deze eicel zich, wat leidt tot de vorming van steeds meer cellen en een toenemende diversiteit tussen deze cellen. Vroeg in de ontwikkeling ontstaan drie belangrijke kiemlijnen: het endoderm, ectoderm en mesoderm. Grof genomen ontwikkelen endodermale cellen zich tot de ingewanden, ectodermale cellen tot de huid en het zenuwstelsel, en mesodermale cellen tot het skelet, bloed en spieren. Hoewel al deze cellen hetzelfde DNA delen, verschillen ze uiteindelijk sterk van elkaar en vervullen ze verschillende taken. Dit alles is mogelijk gemaakt door een sterke regulatie van genexpressie. In dit proefschrift bespreek ik de computationele analyse van genexpressie, met een focus op transcriptieregulatie.

In \textbf{hoofdstuk 1} geef ik een inleiding op de algemene achtergrond van dit proefschrift. Het biedt een overkoepelend overzicht van alle relevante onderwerpen. Eerst behandel ik de regulatie van genexpressie via transcriptiefactoren, de conditie van chromatine rondom genen, en geef ik een beknopte beschrijving van andere regulatoire mechanismen. Vervolgens bespreek ik hoe genexpressie wordt onderzocht, waarbij zowel het experimentele deel in het laboratorium als de daaropvolgende computationele analyse aan bod komen. Tot slot introduceer ik het vakgebied van evolutionaire ontwikkelingsbiologie (evo-devo). Evolutionair ontwikkelingsbiologen zijn geïnteresseerd in zowel de conservering als de verschillen in genregulatie tijdens de (embryonale) ontwikkeling tussen organismen.

In \textbf{hoofdstuk 2} bespreek ik de huidige stand van zaken rondom de analyse van genregulatoire netwerken (GRNs) en doe ik drie aanbevelingen aan het veld. Mijn eerste aanbeveling betreft het integreren van meerdere informatiebronnen met betrekking tot genregulatie. Huidige GRNs zijn vaak uitsluitend gebaseerd op transcripten, terwijl het aantal transcripten slechts beperkt correleert met het aantal eiwitten van hetzelfde gen. Mijn tweede aanbeveling benadrukt het gebruik van single-cell technologie. Single-cell sequencing biedt een zuiverder signaal dan "bulk" sequencing, wat een potentieel positief effect kan hebben op de GRN-analyse. Als laatste aanbeveling stel ik voor om universele GRN-modellen te ontwikkelen. De huidige netwerken zijn vaak gebaseerd op verschillen tussen twee condities, terwijl uiteindelijk dezelfde set instructies (DNA) wordt gedeeld door alle mogelijke GRNs.

In \textbf{hoofdstuk 3} introduceer ik seq2science, een computationeel hulpmiddel voor de preprocessing van data uit functionele genetica. Het biedt ondersteuning voor zowel de preprocessing van transcriptionele data (RNA-seq) als genomische data (ChIP-/ATAC-seq). Het is geïntegreerd met de meeste publieke databases en genereert een uitgebreid overzichtsrapport aan het einde van het proces.

In \textbf{hoofdstuk 4 en 5} behandel ik de definitie en analyse van het fylogenetische stadium. Het fylogenetische stadium is een fase tijdens de embryonale ontwikkeling waarin embryo's uit hetzelfde fylogenetische stam sterk op elkaar zouden lijken, wat heeft geleid tot het idee dat dit stadium evolutionair gezien geconserveerd is. Recentelijk zijn er kwantitatieve studies uitgevoerd naar zowel de morfologische als moleculaire gelijkenis tussen embryo's. In dit hoofdstuk bespreek ik de methodologische uitdagingen van deze analyses en heranalyseer ik eerdere studies. Zo laat ik onder andere zien dat de temporele conservering binnen soorten vaak vele malen sterker is dan tussen soorten, wat de analyse compliceert. Bovendien weerleg ik het omgekeerde zandlopermodel voor vergelijkingen tussen fylogenetische stammen als een statistisch artefact. Al met al is er weinig kwantitatief bewijs voor het bestaan van het fylogenetische stadium.

In \textbf{hoofdstuk 6} introduceer ik SCEPIA, een computationele methode die ondersteunt bij het schatten van de motiefactiviteit van transcriptiefactoren voor single-cell data. Aangezien transcripten slechts beperkt correleren met het aantal eiwitten van hetzelfde gen, koppelt SCEPIA eerst automatisch de transcriptie-informatie van een cel aan een referentiedatabase van H3K27ac. Door vervolgens op deze database de motiefactiviteit te schatten en deze te correleren met het aantal transcripten van dezelfde transcriptiefactor, levert SCEPIA aanzienlijk nauwkeurigere schattingen dan als je alleen naar transcripten kijken. Daarnaast valideer ik de resultaten van SCEPIA met behulp van een dataset van het menselijk hart.

Ten slotte vat ik in \textbf{hoofdstuk 7} mijn eerdere resultaten samen en bespreek ik ze in een bredere context. Ik benadruk daarbij de noodzaak van een systematische weergave van genregulatoire netwerken. Daarnaast uit ik kritiek op het onderzoek naar het fylogenetische stadium, waarbij ik voorbeelden geef van inconsistent gedrag van zowel onderzoekers als hun resultaten, de schijnbare apathie binnen het veld en het dogmatische geloof in het bestaan van het fylogenetische stadium. Tot slot deel ik mijn zorgen over de huidige stand van de bioinformatica, waarin veel computationele methoden en bestanden gebaseerd zijn op verourderde en simplistische aannames.
